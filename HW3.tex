\documentclass{article}\usepackage[]{graphicx}\usepackage[]{xcolor}
% maxwidth is the original width if it is less than linewidth
% otherwise use linewidth (to make sure the graphics do not exceed the margin)
\makeatletter
\def\maxwidth{ %
  \ifdim\Gin@nat@width>\linewidth
    \linewidth
  \else
    \Gin@nat@width
  \fi
}
\makeatother

\definecolor{fgcolor}{rgb}{0.345, 0.345, 0.345}
\newcommand{\hlnum}[1]{\textcolor[rgb]{0.686,0.059,0.569}{#1}}%
\newcommand{\hlsng}[1]{\textcolor[rgb]{0.192,0.494,0.8}{#1}}%
\newcommand{\hlcom}[1]{\textcolor[rgb]{0.678,0.584,0.686}{\textit{#1}}}%
\newcommand{\hlopt}[1]{\textcolor[rgb]{0,0,0}{#1}}%
\newcommand{\hldef}[1]{\textcolor[rgb]{0.345,0.345,0.345}{#1}}%
\newcommand{\hlkwa}[1]{\textcolor[rgb]{0.161,0.373,0.58}{\textbf{#1}}}%
\newcommand{\hlkwb}[1]{\textcolor[rgb]{0.69,0.353,0.396}{#1}}%
\newcommand{\hlkwc}[1]{\textcolor[rgb]{0.333,0.667,0.333}{#1}}%
\newcommand{\hlkwd}[1]{\textcolor[rgb]{0.737,0.353,0.396}{\textbf{#1}}}%
\let\hlipl\hlkwb

\usepackage{framed}
\makeatletter
\newenvironment{kframe}{%
 \def\at@end@of@kframe{}%
 \ifinner\ifhmode%
  \def\at@end@of@kframe{\end{minipage}}%
  \begin{minipage}{\columnwidth}%
 \fi\fi%
 \def\FrameCommand##1{\hskip\@totalleftmargin \hskip-\fboxsep
 \colorbox{shadecolor}{##1}\hskip-\fboxsep
     % There is no \\@totalrightmargin, so:
     \hskip-\linewidth \hskip-\@totalleftmargin \hskip\columnwidth}%
 \MakeFramed {\advance\hsize-\width
   \@totalleftmargin\z@ \linewidth\hsize
   \@setminipage}}%
 {\par\unskip\endMakeFramed%
 \at@end@of@kframe}
\makeatother

\definecolor{shadecolor}{rgb}{.97, .97, .97}
\definecolor{messagecolor}{rgb}{0, 0, 0}
\definecolor{warningcolor}{rgb}{1, 0, 1}
\definecolor{errorcolor}{rgb}{1, 0, 0}
\newenvironment{knitrout}{}{} % an empty environment to be redefined in TeX

\usepackage{alltt}
\usepackage[margin=1.0in]{geometry} % To set margins
\usepackage{amsmath}  % This allows me to use the align functionality.
                      % If you find yourself trying to replicate
                      % something you found online, ensure you're
                      % loading the necessary packages!
\usepackage{amsfonts} % Math font
\usepackage{fancyvrb}
\usepackage{hyperref} % For including hyperlinks
\usepackage[shortlabels]{enumitem}% For enumerated lists with labels specified
                                  % We had to run tlmgr_install("enumitem") in R
\usepackage{float}    % For telling R where to put a table/figure
\usepackage{natbib}        %For the bibliography
\bibliographystyle{apalike}%For the bibliography
\IfFileExists{upquote.sty}{\usepackage{upquote}}{}
\begin{document}

\begin{enumerate}
%%%%%%%%%%%%%%%%%%%%%%%%%%%%%%%%%%%%%%%%%%%%%%%%%%%%%%%%%%%%%%%%%%%%%%%%%%%%%%%%
%%%%%%%%%%%%%%%%%%%%%%%%%%%%%%%%%%%%%%%%%%%%%%%%%%%%%%%%%%%%%%%%%%%%%%%%%%%%%%%%
% QUESTION 1
%%%%%%%%%%%%%%%%%%%%%%%%%%%%%%%%%%%%%%%%%%%%%%%%%%%%%%%%%%%%%%%%%%%%%%%%%%%%%%%%
%%%%%%%%%%%%%%%%%%%%%%%%%%%%%%%%%%%%%%%%%%%%%%%%%%%%%%%%%%%%%%%%%%%%%%%%%%%%%%%%
\item This week's Problem of the Week in Math is described as follows:
\begin{quotation}
  \textit{There are thirty positive integers less than 100 that share a certain 
  property. Your friend, Blake, wrote them down in the table to the left. But 
  Blake made a mistake! One of the numbers listed is wrong and should be replaced 
  with another. Which number is incorrect, what should it be replaced with, and 
  why?}
\end{quotation}
The numbers are listed below.
\begin{center}
  \begin{tabular}{ccccc}
    6 & 10 & 14 & 15 & 21\\
    22 & 26 & 33 & 34 & 35\\
    38 & 39 & 46 & 51 & 55\\
    57 & 58 & 62 & 65 & 69\\
    75 & 77 & 82 & 85 & 86\\
    87 & 91 & 93 & 94 & 95
  \end{tabular}
\end{center}
Use the fact that the ``certain'' property is that these numbers are all supposed
to be the product of \emph{unique} prime numbers to find and fix the mistake that
Blake made.\\
\textbf{Reminder:} Code your solution in an \texttt{R} script and copy it over
to this \texttt{.Rnw} file.\\
\textbf{Hint:} You may find the \verb|%in%| operator and the \verb|setdiff()| function to be helpful.\\

\textbf{Solution:} The incorrect number contained within the table is 75, and should be replaced with the number 74. To come to this conclusion, I first established a vector with the numbers listed above. After such, I created an empty dataframe that would list each number and their respective factors, and filled it with data by creating a function which would output all prime factors for a given value. I then called this function within a loop to determine the factors of each number, and after observing all sets of factors came to the conclusion that 75 was the only listed number which had more than two prime factors. Assuming this was the incorrect number, I instead checked values around 75 to find that 74 had two prime factors that had not appeared within the dataframe previously. Following such, I removed the incorrect values of 75 and its factors, and replaced them with the correct values from 74.


\begin{knitrout}\scriptsize
\definecolor{shadecolor}{rgb}{0.969, 0.969, 0.969}\color{fgcolor}\begin{kframe}
\begin{alltt}
\hldef{numbers} \hlkwb{<-} \hlkwd{c}\hldef{(}\hlnum{6}\hldef{,} \hlnum{10}\hldef{,} \hlnum{14}\hldef{,} \hlnum{15}\hldef{,} \hlnum{21}\hldef{,} \hlcom{#establishes a vector containing the numbers referenced within the problem}
             \hlnum{22}\hldef{,} \hlnum{26}\hldef{,} \hlnum{33}\hldef{,} \hlnum{34}\hldef{,} \hlnum{35}\hldef{,}
             \hlnum{38}\hldef{,} \hlnum{39}\hldef{,} \hlnum{46}\hldef{,} \hlnum{51}\hldef{,} \hlnum{55}\hldef{,}
             \hlnum{57}\hldef{,} \hlnum{58}\hldef{,} \hlnum{62}\hldef{,} \hlnum{65}\hldef{,} \hlnum{69}\hldef{,}
             \hlnum{75}\hldef{,} \hlnum{77}\hldef{,} \hlnum{82}\hldef{,} \hlnum{85}\hldef{,} \hlnum{86}\hldef{,}
             \hlnum{87}\hldef{,} \hlnum{91}\hldef{,} \hlnum{93}\hldef{,} \hlnum{94}\hldef{,} \hlnum{95}\hldef{)}
\hldef{factors_df} \hlkwb{<-} \hlkwd{data.frame}\hldef{(}\hlkwc{Number} \hldef{=} \hlkwd{numeric}\hldef{(),} \hlcom{#stores prime factors}
                 \hlkwc{factor_1} \hldef{=} \hlkwd{numeric}\hldef{(),}
                 \hlkwc{factor_2} \hldef{=} \hlkwd{numeric}\hldef{(),}
                 \hlkwc{factor_3} \hldef{=} \hlkwd{numeric} \hldef{(),}
                 \hlkwc{stringsAsFactors}\hldef{=}\hlnum{FALSE}\hldef{)}

\hldef{prime_factors_Loop} \hlkwb{<-} \hlkwa{function}\hldef{(}\hlkwc{x}\hldef{) \{} \hlcom{#outputs prime factors of x}
  \hldef{factors_for_x} \hlkwb{<-} \hlkwd{c}\hldef{()}   \hlcom{# Vector that will store factors for the value of x}
  \hldef{i} \hlkwb{<-} \hlnum{2}           \hlcom{# Start checking factors beginning two}

  \hlkwa{while} \hldef{(x} \hlopt{>=} \hldef{i) \{}  \hlcom{# Continue looping while x is greater than or equal to i}
    \hlkwa{if} \hldef{(x} \hlopt \hldef{i} \hlopt{==} \hlnum{0}\hldef{) \{}   \hlcom{# If i is a factor of x }
    \hldef{factors_for_x} \hlkwb{<-} \hlkwd{c}\hldef{(factors_for_x, i)}  \hlcom{# Add i to the list of factors}
      \hldef{x} \hlkwb{<-} \hldef{x} \hlopt{/} \hldef{i}  \hlcom{# Divide x by i to remove that factor}
    \hldef{\}} \hlkwa{else} \hldef{\{}
      \hldef{i} \hlkwb{<-} \hldef{i} \hlopt{+} \hlnum{1}  \hlcom{#if i is not a factor, move to the next number}
    \hldef{\}}
  \hldef{\}}
 \hlkwd{return}\hldef{(factors_for_x)} \hlcom{#returns the vector containing the prime factors for x}
\hldef{\}}
\hlkwa{for}\hldef{(j} \hlkwa{in} \hlnum{1}\hlopt{:}\hlkwd{length}\hldef{(numbers))\{}\hlcom{#loop that stores each number and their respective factors in factors_df }
  \hldef{x} \hlkwb{=} \hldef{numbers[j]}
  \hldef{factors_for_x} \hlkwb{<-} \hlkwd{prime_factors_Loop}\hldef{(x)}
\hldef{new_row} \hlkwb{<-} \hlkwd{data.frame}\hldef{(}\hlkwc{Number} \hldef{= numbers[j],}
                      \hlkwc{factor_1} \hldef{= factors_for_x[}\hlnum{1}\hldef{],}
                      \hlkwc{factor_2} \hldef{= factors_for_x[}\hlnum{2}\hldef{],}
                      \hlkwc{factor_3} \hldef{= factors_for_x[}\hlnum{3}\hldef{],}
                      \hlkwc{stringsAsFactors} \hldef{=} \hlnum{FALSE}\hldef{)}

\hldef{factors_df} \hlkwb{<-} \hlkwd{rbind}\hldef{(factors_df, new_row)}  \hlcom{#data frame containing each number within the vector and their prime factors}
\hldef{\}}
\hlcom{#After examining data frame, 75 is the only number with more than two factors, and therefore is the incorrect number}
\hldef{factors_for_74} \hlkwb{<-} \hlkwd{prime_factors_Loop}\hldef{(}\hlnum{74}\hldef{)} \hlcom{#Runs the loop to check factors for 74 instead, which has 2 prime factors}
\hlcom{#After examining the dataframe, we've observed 75's position as 21}
\hldef{factors_df}\hlopt{$}\hldef{Number[}\hlnum{21}\hldef{]} \hlkwb{<-} \hlnum{74} \hlcom{#replaces 75 with the correct number, 74}
\hldef{factors_df}\hlopt{$}\hldef{factor_1[}\hlnum{21}\hldef{]} \hlkwb{<-} \hldef{factors_for_74[}\hlnum{1}\hldef{]} \hlcom{#Replaces factor 1 for 74}
\hldef{factors_df}\hlopt{$}\hldef{factor_2[}\hlnum{21}\hldef{]} \hlkwb{<-} \hldef{factors_for_74[}\hlnum{2}\hldef{]} \hlcom{#Replaces factor 2 for 74}
\hldef{factors_df}\hlopt{$}\hldef{factor_3[}\hlnum{21}\hldef{]} \hlkwb{<-} \hlnum{NA} \hlcom{#Eliminates factor 3 as no numbers have more than two factors}
\end{alltt}
\end{kframe}
\end{knitrout}
\end{enumerate}
\bibliography{bibliography}
\end{document}
